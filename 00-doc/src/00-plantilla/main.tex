% !TeX root = ..\ejemplo.tex


%%%%%%%% PREÁMBULO %%%%%%%%%%%%

\title{Plantilla para prácticas de UPIITA}
\usepackage{preambulo-plantilla}

%%%%%%%% TERMINA PREÁMBULO %%%%%%%%%%%%

\begin{document}

%%%%%%%%%%%%%%%%%%%%%%%%%%%%%%%%%% PORTADA %%%%%%%%%%%%%%%%%%%%%%%%%%%%%%%%%%%%%%%%%%%%
%%%
\input{portada}
%%%

%%%%%%%%%%%%%%%%%%%% TERMINA PORTADA %%%%%%%%%%%%%%%%%%%%%%%%%%%%%%%%

\tableofcontents
\newpage

%%  Compilacion segun parametros
%% Idea de: https://tex.stackexchange.com/questions/1492/passing-parameters-to-a-document

\ifdefined\docRequerimientos
  \import{10-requerimientos}{main}
\else
  \ifdefined\docEjemplo
    \import{01-ejemplo}{main}
  \fi
\fi


\end{document}