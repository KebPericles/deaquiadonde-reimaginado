\section{Descripción general}
% En esta secci´on se describen todos aquellos factores que afectan al producto y a sus requisitos. No se describen los requisitos, sino su contexto. Esto permitir´a definir con detalle los requisitos en la secci´on 3, haciendo que sean m´as f´aciles de entender.
% Normalmente, esta secci´on consta de las siguientes subsecciones: Perspectiva del producto, funciones del producto, caracter´ısticas de los usuarios, restricciones, factores que se asumen y futuros requisito

\subsection{Perspectiva del producto}
% Esta subseccion debe relacionar el futuro sistema (producto software) con otros productos. Si el producto es totalemente independiente de otros productos, tambien debe especificarse aquı. Si la ERS define un producto que es parte de un sistema mayor, esta subseccion relacionara los requisitos del sistema mayor con la funcionalidad del producto descrito en la ERS, y se identificaran las interfaces entre el producto mayor y el producto aquı descrito. Se recomienda utilizar diagramas de bloques

En su estado actual, ``DAAD'' es un producto completamente independiente y no está directamente vinculado a otros sistemas o productos. No requiere interacciones con aplicaciones o servicios externos para funcionar de manera efectiva.


\subsection{Funciones del producto}
% TODO Funciones del producto a grandes rasgos

\begin{itemize}

        \item {
              \textbf{Registro de Usuarios:}

              Los usuarios podrán registrarse proporcionando su dirección de correo electrónico y una contraseña.
              }

        \item {

              \textbf{Inicio de Sesión Único:}

              Se proporcionará un sistema de inicio de sesión único para todos los usuarios.
              }

        \item {
              \textbf{Restablecimiento de Contraseñas:}

              Los usuarios podrán restablecer sus contraseñas en caso de olvido.
              }

        \item {
              \textbf{Guardar materias:}

              Los usuarios podrán guardar materias en su perfil para tener un mejor control de sus horarios.
              }

        \item {
              \textbf{Visualización de Horarios de Clases:}

              Los usuarios podrán ver sus horarios de clases en una interfaz fácil de usar después de iniciar sesión.
              El sistema calculará automáticamente y mostrará en qué salón se llevará a cabo cada clase.
              }

        \item {
              \textbf{Gestión de Materias:}

              Los administradores tendrán la capacidad de agregar, modificar y eliminar información sobre las materias, incluyendo nombre de la materia, profesor asignado, grupos y horarios flexibles.
              }

        \item {
              \textbf{Gestión de Salones y Laboratorios:}

              Los administradores podrán gestionar información sobre salones y laboratorios, incluyendo número, capacidad, tipo y disponibilidad en función de horarios preestablecidos.
              }

\end{itemize}

\subsection{Características de los usuarios}
% TODO
\begin{itemize}
        \item Usuarios Comunes: Estudiantes de la universidad que deseen registrarse y utilizar la plataforma para acceder a información sobre los horarios de clase de los salones.
        \item Administradores: Personal autorizado de la universidad que gestionará la información sobre edificios, salones, laboratorios y materias.
\end{itemize}

% \subsection{Restricciones}
% Limitaciones impuestas en los desarrolladores

\subsection{Suposiciones y dependencias}
% TODO Suposiciones que si cambian pueden afectar a los requisitos
Se asume que los datos proporcionados por los administradores y usuarios son precisos y actualizados.

Se asume que los usuarios tienen acceso a un navegador web moderno y una conexión a Internet.

Se asume que los administradores darán soporte continuamente a la plataforma.

\subsection{Requisitos futuros}
% TODO Funcionalidades Excluidas:

\begin{itemize}
        \item La importación automática de horarios desde el sistema SAES de la escuela.
        \item La integración con Google para la creación de cuentas de usuario.

\end{itemize}