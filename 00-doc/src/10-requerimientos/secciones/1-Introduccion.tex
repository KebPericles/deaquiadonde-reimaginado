% !TeX root = ..\..\requerimientos.tex

\section{Introducción}

\subsection{Propósito}
La aplicación tiene como objetivo ayudar a los estudiantes de la universidad a buscar información sobre la disponibilidad de salones y laboratorios, así como detalles sobre la ubicación de estos espacios.

\subsection{Ámbito del sistema}

Nombre del Sistema: \textbf{¿De aquí a donde? (reimaginado)}
Alias: \textbf{DAAD}

\subsubsection{Descripción del Sistema:}

¿De aquí a dónde? (reimaginado) es una aplicación web diseñada para facilitar a los estudiantes de la universidad acceder a información precisa y actualizada sobre sus horarios de clases y la disponibilidad de salones y laboratorios. La plataforma permitirá a los usuarios registrados, tanto estudiantes como administradores, gestionar información relacionada con las materias, horarios y ubicaciones de clases.

\subsubsection{Funcionalidades Incluidas:}

\begin{itemize}

        \item {
              \textbf{Registro de Usuarios:}

              Los usuarios podrán registrarse proporcionando su dirección de correo electrónico y una contraseña.
              }

        \item {

              \textbf{Inicio de Sesión Único:}

              Se proporcionará un sistema de inicio de sesión único para todos los usuarios.
              Restablecimiento de Contraseñas:

              Los usuarios podrán restablecer sus contraseñas en caso de olvido.
              }

        \item {
              \textbf{Visualización de Horarios de Clases:}

              Los usuarios podrán ver sus horarios de clases en una interfaz fácil de usar después de iniciar sesión.
              El sistema calculará automáticamente y mostrará en qué salón se llevará a cabo cada clase.
              }

        \item {
              \textbf{Gestión de Materias:}

              Los administradores tendrán la capacidad de agregar, modificar y eliminar información sobre las materias, incluyendo nombre de la materia, profesor asignado, grupos y horarios flexibles.
              }

        \item {
              \textbf{Gestión de Salones y Laboratorios:}

              Los administradores podrán gestionar información sobre salones y laboratorios, incluyendo número, capacidad, tipo y disponibilidad en función de horarios preestablecidos.
              }

\end{itemize}

\subsubsection{Funcionalidades Futuras:}

\begin{itemize}
        \item {
              Integración con SAES:

              Se considerará la posibilidad de integrar Reimaginado con el sistema SAES de la universidad para permitir la importación automática de horarios de clases en el futuro.
              }
\end{itemize}

\subsubsection{Beneficios, Objetivos y Metas:}

El sistema tiene como objetivo principal proporcionar una experiencia sin complicaciones a los estudiantes al acceder a información relevante sobre sus clases y las instalaciones disponibles. Los beneficios esperados incluyen una mejor organización de horarios, mayor eficiencia en la gestión de espacios y una experiencia de usuario mejorada para todos los usuarios.


% \subsection{Definiciones, acrónimos y abreviaturas}
% TODO: agregar definiciones, acronimos y abreviaturas (si es necesario)

% TODO: Grupo
% TODO: UPIITA
% TODO: SAES


% \subsection{Referencias}
% TODO: agregar referencias (si es necesario)


% \subsection{Visión general del documento}
% TODO: explicacion de la estructura del resto del documento