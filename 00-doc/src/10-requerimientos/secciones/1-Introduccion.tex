% !TeX root = ..\..\requerimientos.tex

\section{Introducción}

\subsection{Propósito}
La aplicación tiene como objetivo ayudar a los estudiantes de la universidad a buscar información sobre la disponibilidad de salones y laboratorios, así como detalles sobre la ubicación de estos espacios.

\subsection{Ámbito del sistema}
En esta subsección:
Se podr´a dar un nombre al futuro sistema (p.ej. MiSistema)
Se explicar´a lo que el sistema har´a y lo que no har´a.
Se describir´an los beneficios, objetivos y metas que se espera alcanzar
con el futuro sistema.
Se referenciar´an todos aquellos documentos de nivel superior (p.e. en Ingenier´ıa de Sistemas, que incluyen Hardware y Software, deber´ıa mantenerse la consistencia con el documento de especificaci´on de requisitos
globales del sistema, si existe).

Descripción del Sistema:

La aplicación web tiene como objetivo principal proporcionar a los estudiantes de la universidad una plataforma para acceder a información relacionada con la disponibilidad de salones y laboratorios, así como detalles sobre la ubicación de estos espacios. El sistema permitirá a los usuarios registrados visualizar sus horarios de clases y facilitará a los administradores la gestión de la información sobre edificios, salones, laboratorios y materias.

Funcionalidades Incluidas:

Registro y autenticación de usuarios.
Visualización de horarios personales para usuarios registrados.
Gestión de información sobre edificios, incluyendo nombre, ubicación y asociación con salones y laboratorios.
Gestión de información sobre salones y laboratorios, incluyendo número, capacidad, tipo y disponibilidad en función de horarios preestablecidos.
Gestión de información detallada sobre materias, incluyendo nombre de la materia, profesor asignado, grupos y horarios flexibles.
Cálculo automático de horarios de clases basado en la información de materias ingresada por usuarios registrados.
Funcionalidades Excluidas:

La importación automática de horarios desde el sistema SAES de la escuela (esto se considerará como una mejora futura).
Límites del Sistema:

El sistema estará disponible exclusivamente como una aplicación web y no incluirá aplicaciones móviles nativas en este momento.
El sistema se ejecutará en servidores específicos de la universidad y estará sujeto a las políticas de seguridad y recursos de la institución.
Interacciones con Otros Sistemas:

El sistema interactuará con una base de datos interna para almacenar y recuperar datos relevantes.
En el futuro, se considerará la posibilidad de integrar el sistema con SAES para la importación automática de horarios de clases.
Usuarios Involucrados:

Usuarios Comunes: Estudiantes de la universidad que deseen registrarse y utilizar la plataforma para acceder a información sobre sus horarios de clases.
Administradores: Personal autorizado de la universidad que gestionará la información sobre edificios, salones, laboratorios y materias.
Criterios de Éxito:

El sistema se considerará exitoso si cumple con los siguientes criterios:

Los usuarios pueden registrarse y utilizar el sistema de manera efectiva.
Los horarios de clases se calculan correctamente para los usuarios registrados.
Los administradores pueden gestionar con éxito la información relacionada con edificios, salones, laboratorios y materias.
Suposiciones y Restricciones:

Se asume que los datos proporcionados por los administradores y usuarios son precisos y actualizados.
La integración con el sistema SAES está sujeta a futuras consideraciones y recursos disponibles.



\subsection{Definiciones, acrónimos y abreviaturas}
% TODO: agregar definiciones, acronimos y abreviaturas (si es necesario)

% TODO: Grupo


\subsection{Referencias}
% TODO: agregar referencias (si es necesario)

\subsection{Visión general del documento}
% TODO: explicacion de la estructura del resto del documento