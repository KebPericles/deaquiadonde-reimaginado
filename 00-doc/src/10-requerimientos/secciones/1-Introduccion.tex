% !TeX root = ..\..\requerimientos.tex

\section{Introducción}

\subsection{Propósito}
La aplicación tiene como objetivo ayudar a los estudiantes de la universidad a buscar información sobre la disponibilidad de salones y laboratorios, así como detalles sobre la ubicación de estos espacios.

\subsection{Ámbito del sistema}

Nombre del Sistema: \textbf{¿De aquí a donde? (reimaginado)}

Alias: \textbf{DAAD}

\subsubsection{Descripción del Sistema:}

"DAAD" es una aplicación web diseñada para facilitar a los estudiantes de la universidad UPIITA acceder a información precisa y actualizada sobre sus horarios de clases y la disponibilidad de salones y laboratorios. La plataforma permitirá a los administradores, gestionar información relacionada con las materias, horarios y ubicaciones de clases.


\subsubsection{Beneficios, Objetivos y Metas:}

El sistema tiene como objetivo principal proporcionar una experiencia sin complicaciones a los estudiantes al acceder a información relevante sobre sus clases y las instalaciones disponibles. Los beneficios esperados incluyen una mejor organización y mayor difusión sobre los horarios de las instalaciones.


% \subsection{Definiciones, acrónimos y abreviaturas}
% TODO: agregar definiciones, acronimos y abreviaturas (si es necesario)

% TODO: Grupo
% TODO: UPIITA
% TODO: SAES
% TODO: Horarios flexibles
% TODO: Horario de clases


% \subsection{Referencias}
% TODO: agregar referencias (si es necesario)


% \subsection{Visión general del documento}
% TODO: explicacion de la estructura del resto del documento